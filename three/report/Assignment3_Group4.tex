\documentclass{article}
\usepackage[margin=1in]{geometry}
\usepackage{amsmath,amssymb}
\usepackage{verbatim}
\usepackage{graphicx}
\usepackage{xcolor,colortbl}
\usepackage[]{algorithm2e}
%\usepackage{cite}
\usepackage{caption}
\usepackage{mdframed}
\usepackage{float}
\setcounter{tocdepth}{5}
%\usepackage{soul}

\usepackage[backend=bibtex]{biblatex}
\addbibresource{bibliography.bib}


\newcommand{\tab}{\hspace{10mm}}
\newcommand{\dtab}{\hspace{20mm}}
\newcommand{\ttab}{\hspace{30mm}}
\newcommand{\qtab}{\hspace{40mm}}

\begin{document}

\title{Autonomous Agents 1 \\ Assignment 3}

\author{By Group 4: Gieske, Gornishka, Koster, Loor}
\maketitle

\pagebreak
\tableofcontents


\pagebreak

\section{Introduction}
This report contains the analysis of different multi-agent reinforcement learning algorithms. \\
The algorithms were tested in an $11\times11$ grid-world with one prey-agent and $n\in \{1,2,3,4\}$ predator-agents. The goal of the prey is to have the predators bump into each other, while the goal of the predators is to have one of them catch the prey. All agents learn, by receiving a reward of $10$ if their team wins, and a reward of $-10$ if their team loses. \\
Three model-free learning algorithms were compared: Independent Q-learning, Independent Sarsa and Minimax-Q \cite{Littman94markovgames}. Contrary to earlier implementations, the problem at hand considers an intelligent prey, that learns alongside the predators. The effect of the different algorithms, with different parameter settings (learning rate $\alpha$, discount factor $\gamma$, explore-rate $\epsilon$), on different numbers of learning agents was analyzed and the results reported in this paper.


\pagebreak

\section{Theory}

\subsection{Independent Q-learning}
In independent Q-learning, all agents independently use Q-learning to maximize their expected reward.  \\ 

\noindent Q-learning\footnote{More specifically, this is \textit{one-step Q-learning}, which is the algorithm evaluated in this paper.} itself is a temporal difference method that uses the update rule in equation \eqref{eq:qupdate}. Since the algorithm retrieves the Q-value of the state-action pair where Q(s', a) is maximized, this is an \textit{off-policy} method. The algorithm for Q-learning can be found in pseudocode in figure \ref{alg:qlearning}.

\begin{mdframed}
\begin{align}
Q(s_t, a_t) \leftarrow Q(s_t,a_t) + \alpha \left[ r_{t+1} + \gamma \underset{a}{\text{max}} Q(s_{t+1},a) - Q(s_t,a_t)\right]\label{eq:qupdate}
\end{align}
\end{mdframed}


\begin{center} 
\begin{mdframed}
\begin{algorithm}[H]
Initialize Q(s,a) arbitrarily \\
Repeat (for each episode):\\
\tab Initialize s \\
\tab Repeat (for each step of episode):\\
\dtab Choose a from s' using policy derived from Q (e.g., $\epsilon$-greedy)\\
\dtab Take action a, observe r, s'\\
\dtab Q(s,a) $\leftarrow$ Q(s,a) + $\alpha [ r + \gamma \max_a' Q(s', a') - Q(s, a) ]$  \\
\dtab s $\leftarrow$ s'; \\
\tab until s is terminal\\
\end{algorithm}
\end{mdframed}
\captionof{figure}{The algorithm for one-step Q-learning\cite{bartosutton}}
\label{alg:qlearning}
\end{center}


\subsection{Independent Sarsa}
In independent SARSA, all the agents independently use SARSA to maximize their expected reward.  \\ 

\noindent SARSA is a temporal difference method, like Q-learning, that uses the update rule in equation \ref{eq:supdate}. Because this algorithm retrieves the Q value after taking an $\epsilon$-greedy action, it is an \textit{on-policy} method. The algorithm for SARSA can be found in pseudocode in figure \ref{alg:slearning}.

\begin{mdframed}
\begin{align}
Q(s_t, a_t) \leftarrow Q(s_t,a_t) + \alpha \left[ r_{t+1} + \gamma Q(s_{t+1},a_{t+1}) - Q(s_t,a_t)\right]\label{eq:supdate}
\end{align}
\end{mdframed}


\begin{center}
\begin{mdframed}
\begin{algorithm}[H]
Initialize Q(s,a) arbitrarily\\
Repeat (for each episode):\\
\tab Initialize s \\
\tab Choose a from s' using policy derived from Q (e.g., $\epsilon$-greedy)\\
\tab Repeat (for each step of episode):\\
\dtab Take action a, observe r, s'\\
\dtab Choose a' from s' using policy derived from Q (e.g., $\epsilon$-greedy)\\
\dtab Q(s,a) $\leftarrow$ Q(s,a) + $\alpha [ r + \gamma Q(s', a') - Q(s, a) ]$ \\
\dtab s $\leftarrow$ s'; \\
\tab until s is terminal\\
\end{algorithm}
\end{mdframed}
\captionof{figure}{The algorithm for Sarsa \cite{bartosutton}}
\label{alg:slearning}
\end{center}


\subsection{Minimax Q-Learning} 
The problem with independent Q-learning is that it is based on a stationary environment, i.e. where the rules stay the same. However, in a Multi-agent environment where other agents learn as well, the environment is dynamic. This means that the guarantees that hold for single-agent learning do not hold in this setting. Littman \cite{Littman94markovgames} specifically considers two-player zero-sum games. In this type of Markov Game, it is possible to use a single reward function, that one player tries to \textit{maximize}, and the other (the opponent) tries to \textit{minimize}. For MDPs, there is a policy $\pi$ that is optimal. However, for Markov Games, there is often no \textit{undominated} policy\footnote{If a policy is dominated, that means a better policy exists}. The solution to this is to pick a policy and estimate its value by assuming the opponent will take the actions that are worst for the agents, with regards to this policy. In short, minimax picks the policy that maximizes the agent's reward in the worst case. This optimal policy can be stochastic, as seen in the policy for \textit{rock, paper, scissors} (where the best policy is being unpredictable so you cannot be exploited). To find the optimal policy $\pi^*$, linear programming can be used, where the value of a state is
\begin{mdframed}
\begin{align}
V(s) &= \underset{\pi \in PD(A)}{\text{max}} \underset{o \in O}{\text{min}} \sum_{a\in A} Q(s,a,o) \pi_a
\end{align}
\end{mdframed}
and the best action is selected using the Q-values, computed by
\begin{mdframed}
\begin{align}
Q(s,a,o) &= R(s,a,o) + \gamma \sum_{s'} T(s,a,o,s') V(s')
\end{align}
\end{mdframed}
where T is the transition function for transitioning from state $s$ to state $s'$ if the agent picks action $a$ and the opponent picks action $o$. However, since $s$ followed by $s'$ after actions $a$ and $o$ happens with a probability $T(s,a,o,s')$, this function can be left out of the equation.\\

Consequently, each agent uses the following update rule:
\begin{mdframed}
\begin{align}
Q(s,a,o) \leftarrow Q(s,a,o) + \alpha (R + \gamma V(s') - Q(s,a,o))
\end{align}
\label{ref:minimaxrule}
\end{mdframed}
Then, linear programming is used to find a policy $\pi$ so that
\begin{mdframed}
\begin{align}
\pi(s,) \leftarrow \underset{\pi'(s,)}{\text{argmax }} \left\{ \underset{o'}{\text{min}} \left\{ \sum_{a'}  \left\{ \pi(s,a') \times Q(s,a',o') \right\} \right\} \right\}
\end{align}
\end{mdframed}



\begin{center} 
\begin{mdframed}
\begin{algorithm}[H]
Initialize Q(s,a) arbitrarily \\
Repeat (for each episode):\\
\tab Initialize s \\
\tab Repeat (for each step of episode):\\
\dtab Choose a from s' using policy derived from Q (e.g., $\epsilon$-greedy)\\
\dtab Take action a, observe reward R, s' and opponent's action o\\
\dtab Q(s,a,o) $\leftarrow$ Q(s,a,o) + $\alpha $(R + $\gamma$ V(s') - Q(s,a,o))  \\
\dtab $\pi(s,) \leftarrow \underset{\pi'(s,)}{\text{argmax }} \left\{ \underset{o'}{\text{min}} \left\{ \sum_{a'}  \left\{ \pi(s,a') \times Q(s,a',o') \right\} \right\} \right\}$ \\
\dtab $ V(s) \leftarrow \underset{o'}{\text{min}} \left\{ \sum_{a'}  \left\{ \pi(s,a') \times Q(s,a',o') \right\} \right\}  $ \\
\dtab $\alpha \leftarrow \alpha \times decay$
\end{algorithm}
\end{mdframed}
\captionof{figure}{Minimax-Q learning\cite{Littman94markovgames}}
\label{alg:minmax}
\end{center}




\subsection{Policy Hill Climbing} 
Q-learning can be extended to play mixed strategies. Essentially, the policy is improved by increasing the probability of selecting the highest Q-value, according to the learning rate $\delta$. This is called a mixed strategy. Due to the strong relation with Q-learning and its convergence, it is assumed that this algorithm also converges to an optimal policy if all other agents in the game are playing with stationary strategies \cite{bowling2001rational}. This also leads to the policy $\pi$ to converge to a greedy policy, always selecting the highest Q-values. The algorithm is displayed, in pseudocode, in figure \ref{alg:hclearning}.

\begin{center} 
\begin{mdframed}
\begin{algorithm}[H]
Initialize: \\
Q(s,a) $\leftarrow$ 0 \\
$\pi$(s,a) $\leftarrow \dfrac{1}{|\mathcal{A}_i|}$ \\

Repeat: \\
\tab From state \textit{s} select action \textit{a} with probability $\pi(s,a)$ with some exploration \\
\tab Observing reward $\mathcal{r}$ and next state \textit{s'}, \\
\tab $Q(s,a) \leftarrow (1- \alpha)Q(s,a) + \alpha \left( r + \gamma \max\limits_{a'} Q(s',a')\right)$ \\
\tab Update $\pi(s,a)$ and constrain it to a legal probability distribution: \\
$\pi(s,a)\leftarrow \pi (s,a) + \begin{cases}
	\delta, & \text{if } a = \textit{arg} \max_{a'} Q(s,a')\\
	\frac{-\delta}{|\mathcal{A}_i -1|} & \text{otherwise }
\end{cases}$\\
\end{algorithm}
\end{mdframed}
\captionof{figure}{The algorithm for policy hill-climbing \cite{bowling2001rational}}
\label{alg:hclearning}
\end{center}

\begin{comment}
Initialize:
For all s in S, a in A, and o in O,
Let Q[s,a,o] := 1
For all s in S,
Let V[s] := 1
For all s in S, a in A,
Let pi[s,a] := 1/|A|
Let alpha := 1.0
Choose an action:
With probability explor, return an action uniformly at random.
Otherwise, if current state is s,
Return action a with probability pi[s,a].
Learn:
After receiving reward rew for moving from state s to s’
via action a and opponent’s action o,
Let Q[s,a,o] := (1-alpha) * Q[s,a,o] + alpha * (rew + gamma * V[s’])
Use linear programming to find pi[s,.] such that:
pi[s,.] := argmaxfpi’[s,.], minfo’, sumfa’, pi[s,a’] * Q[s,a’,o’]ggg
Let V[s] := minfo’, sumfa’, pi[s,a’] * Q[s,a’,o’]gg
Let alpha := alpha * decay
\end{comment}


% Update which one is the `other leanring'
%\subsection{Friend-or-Foe}
%Friend-or-Foe Q-learning is a multi-agent reinforcement learning technique in which the agent identifies other agents as a 'friend' or a 'foe' . For 'friend' agents a coordination equilibrium is to be found, that is an equilibrium in which all players achieve their highest possible value. If the other agent is a 'foe' agent an adversarial equilibrium is to be found. This equilibrium has the property that the agent is not hurt by any change of the other agent. The algorithm is an adaption of the Nash-Q update rule as seen in equation \ref{eq:nashq}, where the $NashQ^i$ function for agent $i$ is replaced by a function specified by other agents being friend (\ref{eq:friend}) or foe  (\ref{eq:foe}). \\
\begin{mdframed}
\begin{align}
Q_{t+1}^i(s,a^1, \dots, a^n) = (1-\alpha)Q_t^i(,a^1, \dots, a^n )+ \alpha[r_t^i+\gamma NashQ_t^i(s, Q^1, \dots, Q^n)]\label{eq:nashq}
\end{align}
\end{mdframed}

~\\$NashQ^i$ function if the other agents are friends:\\
\begin{mdframed}
\begin{align}
\max\limits_{a_1\in A_1} Q^i[s,a^1, \dots, a^n]\label{eq:friend}
\end{align}
\end{mdframed}

~\\$NashQ^i$ function if the other agents are foes:\\
\begin{mdframed}
\begin{align}
\max\limits_{\pi \in \prod(A_i)} \min\limits_{a_i \in A_i} Q^i[s,a^1, \dots, a^n]\label{eq:foe}
\end{align}
\end{mdframed}



\pagebreak

\section{Implementation}

\subsection{New state space}
In the previous assignments, two agents participated in the game. A predator chased a prey, which hardly ever moved and didn't learn. Now, both the predator and the prey move around on the grid. Both agents learn and it is possible to initialize the game with up to four predators, totaling five agents. Adding an agent to the game increases the state space exponentially, making computation of the algorithms very expensive. Combined with the fact that all algorithms need time to learn and that multiple experiments must be run for accuracy, each iteration in each algorithms will be executed several thousand times. It is, therefore, essential to calculate results as quickly as possible. To reduce the state space complexity, it is encoded as follows:  instead of using the locations of all agents as states, the distance to every other agent is calculated. Along these distances, the Q-values are evaluated and used to take an action. The Q-values will be updated for distances between agents, disregarding the absolute locations of agents and thus, taking advantage of the state space symmetry. Using this technique, the number of states to be updated per agents is reduced, resulting in reducing the state space from $11^{2n}$ to $11^{2(n-1)}$ where $n$ is the number of agents in the grid.

\subsection{Files}

The implementation consists of the following files:
\begin{description}
	\item[Agents\_new] \hfill \\ 
	This file contains implementions of the Agent class, the Prey class and the Predator class. Both the predator and the prey inherit functions of the Agent class. The Agent class contains functions any agent needs, such as a set of actions, a policy and other functions. As the predator is the agent this implementation focuses on, the predator class contains more functions than the predator class.
	
	\item[Helpers] \hfill \\ 
	This file contains many helper functions. These functions aid in computation and decision making, but cannot (and need not) be part of a specific class.
	
	\item[Other\_objects] \hfill \\ % uncertain about policy class description
	This file contains the Policy and Environment classes. The environment of the game as well as the rules are implemented in the Environment class. The Policy class contains the implementation of Q-Learning, Sarsa and $\epsilon$-greedy action selection and more functions that help in determining and optimizing a policy as well as choosing an action of this policy.
	\item[Newstate] \hfill \\ 
	This file contains the Game class as well as a demonstration function. The Game class instantiates the game, initialized the predator and the prey and assigns policies to these. The game is run N times and the result is printed. The demonstration function also performs independent Q-learning, minimax Q-learning and independent Sarsa. It uses $\epsilon$-greedy action selection. The results are printed in the command line and graphs are used for analysis.
\end{description}

\pagebreak


\section{Analysis}
This section discusses the results of the implementations. In order to display and compare results, graphs are used. The title describes which parameters are analysed and the legend shows which color represents which setting of said parameters. Contrary to previous reports, tests were run 5 times with 2000 episodes. Eventually, the results were averaged and used for analysis. In order to analyse the results, several default parameters were established, based on the previous assignments. These are:

\begin{itemize}
\item 2000 runs
\item 5 experiments
\item 2 predators
\item 0.9 discount factor ($\gamma$)
\item 0.5 learning rate ($\alpha$)
\item 0.1 epsilon ($\epsilon$)
\end{itemize}

When testing the implementation or parameters, only the parameter under test changes. All other parameters have the default value as described above.

\subsection{Independent Q-Learning}
This section analyses the effects of independent Q-learning, as well as different parameter settings for this learning method.
\subsubsection{1 predator vs. 1 prey}
To start off, a test was run to see if the new environment behaves as expected. Not much has changed, except allowing multiple predators to be initialized, different functions for Q-learning and policies (they need to be dynamic and change with the number of agents) and the caveat that the prey has to trip during $20\%$ of its actions.

\begin{center}
	\includegraphics[scale=0.3]{1_predator_1_prey_q_learning}
	\captionof{figure}{Independent Q-learning: 1 predator versus 1 prey}
	\label{graph:1vs1}
\end{center}

As the graph in figure \ref{graph:1vs1} shows, the predator learns how to catch the prey more quickly after each time step. However, in the previous implementation, the number of time steps it took the predator to catch the prey became more stable. Though the number of time steps needed to catch the prey drops significantly (apparently, the $20\%$ trip rate more than offsets the intelligence of the prey), there is more variance than before. Also, the algorithm learns more quickly than before. As the state space grows exponentially with every added predator,  the state space was encoded to reduce the amount of possible states and is now of size $11^2$. Therefore, after the predator has spent enough time learning, each greedily chosen action is always in the direction of the prey. This leads to catching the prey quicker than before as there is a much smaller state space to be explored.

\subsubsection{2 predators versus 1 prey}
In this case, there are two predators hunting one prey. The state space is now larger than before ($11^4$ states) and leads to slower computations. Tests have shown that the amount of rounds it takes for any of the predators to catch the prey vary significantly, but more importantly, cannot be informatively represented in a graph. Therefore, the cumulative wins and losses for the predators have been graphed, rather than the amount of rounds it takes the predators to catch the prey. Unsurprisingly,  as shown in figure \ref{graph:2vs1}, the graph for wins becomes steeper, and the graph for losses becomes flat after learning. This indicates that the predators are winning more and losing less. Moreover, the numbers of rounds it takes for the predators to catch the prey improves over time. These numbers have been tracked and logged in table \ref{table:2vs1} below.

\begin{center}
	\includegraphics[scale=0.3]{2_predators_q_learning}
	\captionof{figure}{Independent Q-learning: 2 predators vs. 1 prey}
	\label{graph:2vs1}
\end{center}

%The graph shows that the predators learn to cooperate and catch the prey. This number increases, as the number of predator losses increases more slowly over time, while the number of predator wins increases more quickly over time. As this is counted cumulatively, it can be expected that the number of wins by the prey will become almost steady. 

It is expected that, in the end, the predators will win almost each game. As the policies of the predators and the prey are still exploratory, it is possible for two predators to bump into one another and lose the game.

The table \ref{table:2vs1} shows the average number of rounds the predators need to catch the prey.

\begin{table}[H]
\begin{center}
\begin{tabular}{| l | l | l | l | l |}
\hline
 & \parbox{2cm}{\textbf{Avg wins \\ (first 100)}} & \parbox{2cm}{\textbf{Avg losses \\ (first 100)}} & \parbox{2cm}{\textbf{Avg wins \\ (last 100)}} & \parbox{2cm}{\textbf{Avg losses \\ (last 100)}} \\
\hline
\textbf{Predators} & 58 & 42 & 76 & 23 \\
\hline
\end{tabular}
\caption{Average number of rounds two predators need to catch one prey}
\label{table:2vs1}
\end{center}
\end{table}

As the table shows, the predators learn to catch the prey quicker over time.

\subsubsection{3 predators versus 1 prey}
With four agents on the grid, the implementation became very slow. It was possible to run the implementation, but as it is very slow, the parameter changes have not been tested. However, figure \ref{graph:3vs1} shows the results of running three predators versus one prey. Due to time constraints and the amount of time it takes for the algorithm to calculate the all the states, this was for 1000 episodes. Contrary to the other testing approaches, this test was run once and so not averaged over several experiments.

\begin{center}
	\includegraphics[scale=0.3]{3_predators_3_ex_1000_runs_q-learning}
	\captionof{figure}{Independent Q-learning: 3 predators vs. 1 prey}
	\label{graph:3vs1}
\end{center}


\newcommand{\labelerrorfootnote}{\footnote{Due to a labeling error during test execution, the title which appears on the figure itself does not match the one under the figure. The one under the figure is the correct one and the figure does reflect the results for 3 predators vs. 1 prey.}}

Figure \ref{graph:3vs1} shows that the predators lose the game a lot.\labelerrorfootnote \: This is interesting as it is expected of the predators to learn not to bump into one another. However, the grid is both toroidal as well as small and the prey learns. It is possible that the prey learning, combined with a small, toroidal grid, leads to the predators bumping into one another. Perhaps the prey learns to trick the predators into bumping into one another. Therefore, it is interesting to see what happens if the prey learns slower than the predators. By making the prey learn slower, theoretically it is possible for the predators to learn not to bump into one another and catch the prey. In order to simulate this, the predators learn as stated in default, but the prey learns with a learning rate of 0.01. These results are shown in figure \ref{graph:3vs1slowprey}.

\begin{center}
	\includegraphics[scale=0.3]{smaller_learning_rate_prey}
	\captionof{figure}{Independent Q-learning: 3 predators vs. 1 prey, slow learning prey}
	\label{graph:3vs1slowprey}
\end{center}

Table \ref{table:3vs1} shows the results of the first- and last 100 wins and losses of the predators.

\begin{table}[H]
\begin{center}
\begin{tabular}{| l | l | l | l | l |}
\hline
 & \parbox{2cm}{\textbf{Avg wins \\ (first 100)}} & \parbox{2cm}{\textbf{Avg losses \\ (first 100)}} & \parbox{2cm}{\textbf{Avg wins \\ (last 100)}} & \parbox{2cm}{\textbf{Avg losses \\ (last 100)}} \\
\hline
\textbf{Default learning rate} & 21 & 78 & 25 & 73 \\
\hline
\textbf{Low learning rate} & 23 & 76 & 31 & 68 \\
\hline
\end{tabular}
\caption{Average number of wins and losses by the predators with varying learning rates}
\label{table:3vs1}
\end{center}
\end{table}

The results in table \ref{table:3vs1} show that the predators do learn to catch the prey. When the prey learns slower, the predators manage to catch the prey more often in the same amount of time. Therefore, it can be concluded that the algorithm does learn the predators to cooperate and catch the prey. This just takes many episodes.

\subsubsection{4 predators versus 1 prey}
Though it is implemented for four predators and one prey to be placed on the grid, this leads to implementations freezing. It is therefore conclusive to state that the program has become intractable. This could be solved by using function approximation, for example using Kanerva coding \cite{wu2009function}, where a number of \textit{prototype} state-action pairs are selected, to be used for storing Q-values for similar state-action pairs. Another possible solution is to have the predators learn \textit{together}, meaning they share a Q-value grid and learn simultaneously. However, this is not truly independent Q-learning. 

\subsubsection{Parameter settings}
It is interesting to see what happens when the parameters of the learning methods change. As the effects of parameter settings have been researched in a 1 vs. 1 scenario, it is interesting to see what is different when there are more agents on the grid. Also, as all agents now learn, the effects of these learning methods should change.

\subsubsection{Learning rate}
First, the effect of the learning rate is researched. As the learning rate determines to what extent the newly acquired information will override the old information, it is interesting to see what happens.

\begin{center}
	\includegraphics[scale=0.3]{2_predators_learning_rate_q_learning}
	\captionof{figure}{Independent Q-learning: 2 predators vs. 1 prey, learning rate}
	\label{graph:learningrate}
\end{center}

Figure \ref{graph:learningrate} shows that a low learning rate yields worst results. For a long time, a high learning rate yields good results, however, in the end a learning rate of 0.5 yields best results. This shows that for a long time, a lot of recent information is interesting.  Later on, however, an even balance of new and old information leads to more wins for the predator. 

\begin{table}[H]
\begin{center}
\begin{tabular}{| l | l | l | l | l |}
\hline
\parbox{2cm}{\textbf{Learning rate}} & \parbox{2cm}{\textbf{Avg wins \\ (first 100)}} & \parbox{2cm}{\textbf{Avg losses \\ (first 100)}} & \parbox{2cm}{\textbf{Avg wins \\ (last 100)}} & \parbox{2cm}{\textbf{Avg losses \\ (last 100)}} \\
\hline
\textbf{0.2} & 50 & 49 & 74 & 24 \\
\hline
\textbf{0.5} & 54 & 45 & 72 & 27 \\
\hline
\textbf{0.7} & 54 & 45 & 63 & 35 \\
\hline
\end{tabular}
\caption{Average number of wins and losses by the predators with varying learning rates}
\label{table:learningrate}
\end{center}
\end{table}

Table \ref{table:learningrate} shows that the lowest learning rate shows better and better results over time. This shows that in the beginning, a lot must be learned. As the game progresses, a low learning rate yields better results. This could indicate that the predators as well as the prey become predictable and so less new information has to be learned. As minimax Q-learning contains a decay in learning, perhaps this is the reason why.\red{THIS IS INDEED STRANGE!!!! }  %this is strange.  


\subsubsection{Discount factor}
The discount factor determines the importance of future rewards. In the previous assignment, when only one predator was learning, a high discount factor yielded best results. This means that the future reward was most important. Only the goal state yielded a reward, making reaching the goal state very important. Currently, there are two terminal states: the win state and the lose state. It is interesting to see what effect the negative rewards have on the importance of the immediate reward.

\begin{center}
	\includegraphics[scale=0.3]{2_predators_discount_factor_q_learning}
	\captionof{figure}{Independent Q-learning: 2 predators vs. 1 prey, discount factor}
	\label{graph:discountfactor}
\end{center}

Figure \ref{graph:discountfactor} shows that for a long time, it does not matter how important the future reward is. However, eventually the graph shows that a low discount factor yields best results. This might be caused by the fact that the predators will receive a negative reward when running into another predator. In order to avoid this, winning (or not losing) immediately is more important than expected future rewards.

\begin{table}[H]
\begin{center}
\begin{tabular}{| l | l | l | l | l |}
\hline
\parbox{2cm}{\textbf{Discount factor}} & \parbox{2cm}{\textbf{Avg wins \\ (first 100)}} & \parbox{2cm}{\textbf{Avg losses \\ (first 100)}} & \parbox{2cm}{\textbf{Avg wins \\ (last 100)}} & \parbox{2cm}{\textbf{Avg losses \\ (last 100)}} \\
\hline
\textbf{0.2} & 52 & 47 & 98 & 4 \\
\hline
\textbf{0.5} & 55 & 44 & 78 & 21 \\
\hline
\textbf{0.7} & 55 & 45 & 74 & 24 \\
\hline
\end{tabular}
\caption{Average number of predator wins and losses for varying discount factors}
\label{table:discountfactor}
\end{center}
\end{table}

Table \ref{table:discountfactor} shows that the discount factor has a huge impact on the success of the predators. By making sure that the predators do not run into each other, the game is not lost as often.

\subsubsection{$\epsilon$-greedy action selection}
In the current implementation, $\epsilon$-greedy action selection was used to find actions for the agents. For this action-selection mechanism, $\epsilon$ determines the percentage of greedy versus exploratory actions. An $\epsilon$ value of 0 selects only greedy actions. The closer this value is to 1, the more exploring actions are selected. The following figure \ref{graph:greedy} shows the results of this test.

\begin{center}
	\includegraphics[scale=0.3]{2_predators_epsilon_q_learning}
	\captionof{figure}{Independent Q-learning: 2 predators vs. 1 prey, $\epsilon$-greedy action selection}
	\label{graph:greedy}
\end{center}

Figure \ref{graph:greedy} shows that greedy action selection yields best results. This is possible, as all predators are initialized at the corners of the grid, starting out with equal distance to the prey. As the prey moves, it will be closer to one predator. Therefore, after learning and without exploration, the prey will be caught by one predator (as it tries to minimize the distance between the prey and itself). As a greedy action, in this case, leads to moving in the direction of the highest Q-value, it is still possible for the predators to bump into each other. However, it seems as if the prey is most often caught before this happens leading to wins for the predators.

\begin{table}[H]
\begin{center}
\begin{tabular}{| l | l | l | l | l |}
\hline
\parbox{2cm}{\textbf{$\epsilon$-rate}} & \parbox{2cm}{\textbf{Avg wins \\ (first 100)}} & \parbox{2cm}{\textbf{Avg losses \\ (first 100)}} & \parbox{2cm}{\textbf{Avg wins \\ (last 100)}} & \parbox{2cm}{\textbf{Avg losses \\ (last 100)}} \\
\hline
\textbf{0} & 55 & 44 & 76 & 22 \\
\hline
\textbf{0.2} & 54 & 45 & 77 & 21 \\
\hline
\textbf{0.5} & 49 & 50 & 72 & 27 \\
\hline
\textbf{0.7} & 48 & 51 & 66 & 32 \\
\hline
\textbf{0.9} & 50 & 59 & 55 & 44 \\
\hline
\end{tabular}
\caption{Average number of wins and losses by the predators with varying $\epsilon$-rates}
\label{table:greedy}
\end{center}
\end{table}

\red{THERE SEEMS TO BE NO INFO ABOUT TABLE \ref{table:greedy}}	

Though, at first, an absolute greedy policy appears to yield the most promising results, a slightly exploratory policy eventually yields the best results. This is not displayed in the graph as the wins are counted cumulatively. Eventually, the results will be best when still exploring sightly.


\subsection{Independent Sarsa}
This section discusses the effects of independent SARSA learning.
\subsubsection{1 predator vs. 1 prey}
Again, it is interesting to see what happens when there are one predator and one prey playing the game. 

\subsection{Minimax Q-learning}
\pagebreak


\section{Conclusion}
This section discusses the conclusions drawn from the analysis. %This report has discussed independent multi-agent learning as well as minimax learning in as zero-sum game.

\subsubsection{Independent learning}
Independent learning was performed in two different ways: Q-learning and SARSA. Both suck dick, ass and balls when more than 2 predators enter the grid. This is to be expected as the agents learn independently from one another. Though all agents act on what is best for them, they do not work together. Therefore, all agents must learn each others behaviour before being able to catch the prey without bumping into another predator. This takes very long and even then it is not certain that the predators will catch the prey. As all agents learn each others policies, the prey will also learn what the predators will do and may trick them into bumping into one another.

\subsubsection{Minimax Q-learning}
Draw thine conclusions and place them here. Oh Romeo, Romeo. Wherefore art thou Romeo? :'(
\pagebreak


\section{Future work}
This section contains information about improvements in the future

\subsubsection{State space encoding}
It has shown that the state space encoding has improved the performance of the program. However, in order to run tests with four predators on one grid, a smarter encoding is necessary. With a smarter state space encoding, it will be possible to run the program with up to four predators and this research can be completed. The smarter state space encoding will also allow multiple experiments to be run on grids with three or more predators. 
\pagebreak


\section{Files attached}

\begin{itemize}
\item newstate.py
\item agents\_new.py
\item other\_objects.py
\item helpers.py \ldots
\end{itemize}


\section{Sources}

% TODO Update the bibliography according to the new assignment
\nocite{*}
\printbibliography


\begin{comment}
\bibliography{bibliography}
\bibliographystyle{plain}
\begin{itemize}
	\item [1] Barto and Sutton (http://webdocs.cs.ualberta.ca/~sutton/book/the-book.html) \ldots
\end{itemize}
\end{comment}


\end{document}