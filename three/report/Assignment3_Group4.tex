\documentclass{article}
\usepackage[margin=1in]{geometry}
\usepackage{amsmath,amssymb}
\usepackage{verbatim}
\usepackage{graphicx}
\usepackage{xcolor,colortbl}
\usepackage[]{algorithm2e}
%\usepackage{cite}
\usepackage{caption}
\usepackage{mdframed}
\usepackage{float}
%\usepackage{soul}


\usepackage[backend=bibtex]{biblatex}
\addbibresource{bibliography.bib}


\newcommand{\tab}{\hspace{10mm}}
\newcommand{\dtab}{\hspace{20mm}}
\newcommand{\ttab}{\hspace{30mm}}
\newcommand{\qtab}{\hspace{40mm}}

\begin{document}

\title{Autonomous Agents 1 \\ Assignment 3}

\author{By Group 4: Gieske, Gornishka, Koster, Loor}
\maketitle
\pagebreak
\tableofcontents


\pagebreak

\section{Introduction}


\pagebreak

\section{Theory}

\subsection{Independent Q-learning}
In independent Q-learning, all the agents independently use Q-learning to maximize their expected reward.  \\ 

\noindent Q-learning\footnote{More specifically, this is \textit{one-step Q-learning}, which is the algorithm evaluated in this paper.} itself is a temporal difference method that uses the update rule in equation \ref{eq:qupdate}. Because it retrieves the Q-value of the state-action pair where Q(s', a) is maximized, it is an \textit{off-policy} method. The algorithm for Q-learning can be found in pseudocode in figure \ref{alg:qlearning}.

\begin{mdframed}
\begin{align}
Q(s_t, a_t) \leftarrow Q(s_t,a_t) + \alpha \left[ r_{t+1} + \gamma \underset{a}{\text{max}} Q(s_{t+1},a) - Q(s_t,a_t)\right]\label{eq:qupdate}
\end{align}
\end{mdframed}


\begin{center} 
\begin{mdframed}
\begin{algorithm}[H]
Initialize Q(s,a) arbitrarily \\
Repeat (for each episode):\\
\tab Initialize s \\
\tab Repeat (for each step of episode):\\
\dtab Choose a from s' using policy derived from Q (e.g., $\epsilon$-greedy)\\
\dtab Take action a, observe r, s'\\
\dtab Q(s,a) $\leftarrow$ Q(s,a) + $\alpha [ r + \gamma \max_a' Q(s', a') - Q(s, a) ]$  \\
\dtab s $\leftarrow$ s'; \\
\tab until s is terminal\\
\end{algorithm}
\end{mdframed}
\captionof{figure}{The algorithm for one-step Q-learning\cite{bartosutton}}
\label{alg:qlearning}
\end{center}





\subsection{Minimax Q-Learning} 
The problem with independent Q-learning is that it is based on a stationary environment, i.e. where the rules stay the same. However, in a Multi-agent environment where other agents learn as well, the environment is dynamic. This means that the guarantees that hold for single-agent learning do not hold in this setting. Littman \cite{Littman94markovgames} specifically considers two-player zero-sum games. In this type of Markov Game, it is possible to use a single reward function, that one player tries to \textit{maximize}, and the other (the opponent) tries to \textit{minimize}. For MDPs, there is a policy $\pi$ that is optimal. However, for Markov Games, there is often no \textit{undominated} policy\footnote{If a policy is dominated, that means a better policy exists}. The solution to this is to pick a policy and estimate its value by assuming the opponent will take the actions that are worst for the agents, with regards to this policy. In short, minimax picks the policy that maximizes the agent's reward in the worst case. This optimal policy can be stochastic, as seen in the policy for \textit{rock, paper, scissors} (where the best policy is being unpredictable so you cannot be exploited). To find the optimal policy $\pi^*$, linear programming can be used, where the value of a state is
\begin{mdframed}
\begin{align}
V(s) &= \underset{\pi \in PD(A)}{\text{max}} \underset{o \in O}{\text{min}} \sum_{a\in A} Q(s,a,o) \pi_a
\end{align}
\end{mdframed}
and the best action is selected using the Q-values, computed by
\begin{mdframed}
\begin{align}
Q(s,a,o) &= R(s,a,o) + \gamma \sum_{s'} T(s,a,o,s') V(s')
\end{align}
\end{mdframed}
where T is the transition function for transitioning from state $s$ to state $s'$ if the agent picks action $a$ and the opponent picks action $o$. However, since $s$ followed by $s'$ after actions $a$ and $o$ happens with a probability $T(s,a,o,s')$, this function can be left out of the equation.\\

Consequently, each agent uses the following update rule:

\begin{mdframed}
\begin{align}
Q(s,a,o) & \leftarrow Q(s,a,o) + \alpha (R + \gamma V(s') - Q(s,a,o))
\end{align}
\label{ref:minimaxrule}
\end{mdframed}
Then, linear programming is used to find a policy $\pi$ so that
\begin{mdframed}
\begin{align}
\pi(s,) \leftarrow \underset{\pi'(s,)}{\text{argmax }} \left\{ \underset{o'}{\text{min}} \left\{ \sum_{a'}  \left\{ \pi(s,a') \times Q(s,a',o') \right\} \right\} \right\}
\end{align}
\end{mdframed}

Eventually, the update rules can be used to implement the minimax-Q algorithm as described in figure \ref{alg:minmax}.

\begin{center} 
\begin{mdframed}
\begin{algorithm}[H]
Initialize Q(s,a) arbitrarily \\
Repeat (for each episode):\\
\tab Initialize s \\
\tab Repeat (for each step of episode):\\
\dtab Choose a from s' using policy derived from Q (e.g., $\epsilon$-greedy)\\
\dtab Take action a, observe reward R, s' and opponent's action o\\
\dtab Q(s,a,o) $\leftarrow$ Q(s,a,o) + $\alpha $(R + $\gamma$ V(s') - Q(s,a,o))  \\
\dtab $\pi(s,) \leftarrow \underset{\pi'(s,)}{\text{argmax }} \left\{ \underset{o'}{\text{min}} \left\{ \sum_{a'}  \left\{ \pi(s,a') \times Q(s,a',o') \right\} \right\} \right\}$ \\
\dtab $ V(s) \leftarrow \underset{o'}{\text{min}} \left\{ \sum_{a'}  \left\{ \pi(s,a') \times Q(s,a',o') \right\} \right\}  $ \\
\dtab $\alpha \leftarrow \alpha \times decay$
\end{algorithm}
\end{mdframed}
\captionof{figure}{Minimax-Q learning\cite{Littman94markovgames}}
\label{alg:minmax}
\end{center}






\begin{comment}
Initialize:
For all s in S, a in A, and o in O,
Let Q[s,a,o] := 1
For all s in S,
Let V[s] := 1
For all s in S, a in A,
Let pi[s,a] := 1/|A|
Let alpha := 1.0
Choose an action:
With probability explor, return an action uniformly at random.
Otherwise, if current state is s,
Return action a with probability pi[s,a].
Learn:
After receiving reward rew for moving from state s to s’
via action a and opponent’s action o,
Let Q[s,a,o] := (1-alpha) * Q[s,a,o] + alpha * (rew + gamma * V[s’])
Use linear programming to find pi[s,.] such that:
pi[s,.] := argmaxfpi’[s,.], minfo’, sumfa’, pi[s,a’] * Q[s,a’,o’]ggg
Let V[s] := minfo’, sumfa’, pi[s,a’] * Q[s,a’,o’]gg
Let alpha := alpha * decay
\end{comment}


% Update which one is the `other leanring'
\subsection{Friend-or-Foe}
Friend-or-Foe Q-learning is a multi-agent reinforcement learning technique in which the agent identifies other agents as a 'friend' or a 'foe' . For 'friend' agents a coordination equilibrium is to be found, that is an equilibrium in which all players achieve their highest possible value. If the other agent is a 'foe' agent an adversarial equilibrium is to be found. This equilibrium has the property that the agent is not hurt by any change of the other agent. The algorithm is an adaption of the Nash-Q update rule as seen in equation \ref{eq:nashq}, where the $NashQ$ function is replaced by a function specified for friend (\ref{eq:friend}) or foe agents (\ref{eq:foe}). \\
\begin{mdframed}
\begin{align}
Q_{t+1}^j(s,a^1, \dots, a^n) = (1-\alpha)Q_t^j(,a^1, \dots, a^n )+ \alpha[r_t^j+\gamma NashQ_t^j(s, Q^1, \dots, Q^n)]\label{eq:nashq}
\end{align}
\end{mdframed}

If agent is friend:
\begin{mdframed}
\begin{align}
\max\limits_{a_1\in A_1} Q^i[s,a^1, \dots, a^n]\label{eq:friend}\\
\end{align}
\end{mdframed}

%\begin{mdframed}
%\begin{align}
%\max\limits_{\pi \in \product(A_i)} min\limits_{a_i \in A_i} Q^i[s,a^1, \dots, a^n]\ref{eq:foe}\\
%\end{align}
%\end{mdframed}



\section{Implementation}

\subsection{New state space}
In the previous assignments, two agents participated in the game. A predator chased a prey, which hardly ever moved and didn't learn. Now, the both the predator and the prey move around on the grid. Both agents learn and it is possible to initialize the game with up to four predators. This leads to a total of five agents. Adding an agent to the game increases the state space exponentially, making computation of Q-learning very expensive. Combined with the fact that all algorithms need time to learn and that multiple experiments must be run for accuracy, the algorithms will be executed several thousand times. It is, therefore, essential to calculate results as quick as possible. This will speed up significantly using state space encoding. Instead of checking the entire grid per agent, the distance to each agent is calculated. Along these distances, the Q-values are evaluated and used to take an action. The Q-values will be updated, not longer needing to update the entire grid after taking an action. Using this technique, only part of the Q-values are updated, per agent. This saves a lot of time, compared to the previous implementation.

\subsection{Files}

The implementation consists of the following files:
\begin{description}
	\item[Agents\_new] \hfill \\ 
	This file contains implementions of the Agent class, the Prey class and the Predator class. Both the predator and the prey inherit functions of the Agent class. The Agent class contains functions any agent needs, such as a set of actions, a policy and other functions. As the predator is the agent is the agent this implementation focuses on, the predator class contains more functions than the predator class.
	
	\item[Helpers] \hfill \\ 
	This file contains many helper functions. These functions aid in computation and decision making, but cannot (and need not) be part of a specific class.
	
	\item[Other\_objects] \hfill \\ % uncertain about policy class description
	This file contains the Policy and Environment classes. The environment of the game as well as the rules are implemented in the Environment class. The Policy class contains the implementation of Q-Learning, Sarsa, $\epsilon$-greedy, softmax action selection and more functions that help in determining and optimizing a policy as well as choosing an action of this policy.
	\item[Newstate] \hfill \\ 
	This file contains the Game class as well as a demonstration function. The Game class instantiates the game, initialized the predator and the prey and assigns policies to these. The game is run N times and the result is printed. The demonstration function also performs Q-Learning, Sarsa and Monte Carlo. It also uses $\epsilon$-greedy and softmax action selection. The results are printed in the command line and graphs are used for analysis.
\end{description}

\pagebreak


\section{Analysis}
This section discusses the results of the implementations. In order to display and compare results, graphs are used. The title describes which parameters are analysed and the legend shows which color represents which setting of said parameters.

\subsection{Independent Q-Learning}

\subsection{Minmax Q-learning}

\subsection{Other Learning}

\pagebreak


\section{Conclusion}

\pagebreak


\section{Future work}

\pagebreak


\section{Files attached}

\begin{itemize}
\item newstate.py
\item agents\_new.py
\item other\_objects.py
\item helpers.py \ldots
\end{itemize}


\section{Sources}

% TODO Update the bibliography according to the new assignment
\nocite{*}
\printbibliography


\begin{comment}
\bibliography{bibliography}
\bibliographystyle{plain}
\begin{itemize}
	\item [1] Barto and Sutton (http://webdocs.cs.ualberta.ca/~sutton/book/the-book.html) \ldots
\end{itemize}
\end{comment}


\end{document}